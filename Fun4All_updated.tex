\documentclass[12pt]{article}

\usepackage{amsmath}
\usepackage{graphicx}
\usepackage{amssymb}
\usepackage{mathrsfs}
\usepackage{geometry}
\usepackage{color}
\usepackage{verbatimbox}
\geometry{margin=2cm}

\usepackage{url}
\usepackage{tcolorbox}
\usepackage{multirow}
\usepackage{float}

\usepackage{setspace}
\onehalfspacing

\usepackage{hyperref}
\hypersetup{
    colorlinks=true,
    linkcolor=blue,
    filecolor=magenta,      
    urlcolor=cyan,
}

\newcommand\tab[1][1cm]{\hspace*{#1}}

\title{Running Fun4All on Cori (updated)}
\author{Reynier Cruz Torres \\ (borrowing some text from D. Dixit and F. Torales-Acosta)}

\begin{document}

\maketitle

\tableofcontents

% ========================================================================================
\newpage
\section{Logging into Cori and getting Fun4All working}

First, ssh into cori (\verb|ssh -Y username@cori.nersc.gov|) and go to the directory where Fun4All is to be installed.
Clone the Singularity container from Github and install the package following:

\begin{tcolorbox}
\begin{verbnobox}[\scriptsize]
git clone https://github.com/sPHENIX-Collaboration/Singularity.git
cd Singularity/
mkdir install
./updatebuild.sh
shifter --image=docker:ddixit/fun4all:eicresearch /bin/bash
\end{verbnobox}  
\end{tcolorbox}

The last command line above lets you enter the docker image using shifter and sets the shell to bash.

Create a file called \verb|set_g4.sh| containing the following lines:

\begin{tcolorbox}
\begin{verbnobox}[\scriptsize]
export core_dir=/global/path/to/Singularity/cvmfs/sphenix.sdcc.bnl.gov/gcc-8.3/opt/sphenix/core
export G4LEVELGAMMADATA=$core_dir/geant4.10.06.p02/share/Geant4-10.6.2/data/PhotonEvaporation5.5
export G4_MAIN=$core_dir/geant4.10.06.p02
export G4LEDATA=$core_dir/geant4.10.06.p02/share/Geant4-10.6.2/data/G4EMLOW7.9.1
export G4NEUTRONHPDATA=$core_dir/geant4.10.06.p02/share/Geant4-10.6.2/data/G4NDL4.6
export G4ENSDFSTATEDATA=$core_dir/geant4.10.06.p02/share/Geant4-10.6.2/data/G4ENSDFSTATE2.2
export G4RADIOACTIVEDATA=$core_dir/geant4.10.06.p02/share/Geant4-10.6.2/data/RadioactiveDecay5.4
export G4ABLADATA=$core_dir/geant4.10.06.p02/share/Geant4-10.6.2/data/G4ABLA3.1
export G4PIIDATA=$core_dir/geant4.10.06.p02/share/Geant4-10.6.2/data/G4PII1.3
export G4PARTICLEXSDATA=$core_dir/geant4.10.06.p02/share/Geant4-10.6.2/data/G4PARTICLEXS2.1
export G4SAIDXSDATA=$core_dir/geant4.10.06.p02/share/Geant4-10.6.2/data/G4SAIDDATA2.0
export G4REALSURFACEDATA=$core_dir/geant4.10.06.p02/share/Geant4-10.6.2/data/RealSurface2.1.1
export G4INCLDATA=$core_dir/geant4.10.06.p02/share/Geant4-10.6.2/data/G4INCL1.0
\end{verbnobox}  
\end{tcolorbox}

Create another file called $e.g.$ \verb|setup_fun4all.sh| and include the following lines (modify the lines depending on where your files are located):

\begin{tcolorbox}
\begin{verbnobox}[\scriptsize]
singularity_dir="/global/path/to/Singularity"
source ${singularity_dir}/cvmfs/sphenix.sdcc.bnl.gov/gcc-8.3/opt/sphenix/core/bin/sphenix_setup.sh
export MYINSTALL="${singularity_dir}/install" 
source ${singularity_dir}/cvmfs/sphenix.sdcc.bnl.gov/gcc-8.3/opt/sphenix/core/bin/setup_local.sh $INST
source /global/path/to/set_g4.sh
\end{verbnobox}  
\end{tcolorbox}

In principle you should only need to modify the paths in the first and last lines.

\verb|source setup_fun4all.sh|

Check that the setup was successful by running in the terminal: \verb|lsb_release -a|.
You should see the following output:

\begin{tcolorbox}
\begin{verbnobox}[\scriptsize]
LSB Version:	:core-4.1-amd64:core-4.1-ia32:core-4.1-noarch
Distributor ID:	Scientific
Description:	Scientific Linux release 7.3 (Nitrogen)
Release:	7.3
Codename:	Nitrogen
Input: root #see if root works for you.
\end{verbnobox}  
\end{tcolorbox}

At this stage you can also run \verb|root| in the terminal to check that Root is already working.
Once you have root working, try the steps from the ``Try an event display'' tutorial here:

\href{https://github.com/sPHENIX-Collaboration/macros}{https://github.com/sPHENIX-Collaboration/macros}

Remember, we will be using \verb|sphenix_setup.sh| not \verb|.csh|. If the you can run ``Try an event display successfully'', congratulations, you are able to run Fun4All on Cori.
Finally, create a directory called \verb|install| inside the Singularity directory.

\subsection{Using Fun4All from now on (follow these steps every time you log into Cori)}
\label{sec:every_login}

From now on, you will always have to do the following steps to access Fun4All again on Cori:

\begin{tcolorbox}
\begin{verbnobox}[\scriptsize]
shifter --image=docker:ddixit/fun4all:eicresearch /bin/bash
source setup_fun4all.sh
\end{verbnobox}  
\end{tcolorbox}

Step 1 enters the Singularity image, and Step 2 sets up all environment variables.

% ========================================================================================
\newpage
\section{Additional repositories}

\subsection{Building a package}
\label{sec:build}

The steps are explained in detail in the following page in the ``Building a package'' section:

\href{https://wiki.bnl.gov/sPHENIX/index.php/Example_of_using_DST_nodes}{https://wiki.bnl.gov/sPHENIX/index.php/Example\_of\_using\_DST\_nodes}

The basic steps are the following:

\begin{enumerate}
\item Create a build directory. For instance, if the package that will be installed is called ``package'', do \verb|mkdir build_package|.
\item Go into this directory: \verb|cd build_package|.
\item \verb|path/to/package/source/autogen.sh --prefix=$MYINSTALL|.
\item \verb|make install -j 4|
\item Lastly, redo the last three steps from section~\ref{sec:every_login}:

\begin{tcolorbox}
\begin{verbnobox}[\scriptsize]
source local_path_to/opt/sphenix/core/bin/sphenix_setup.sh -n
export MYINSTALL=/global/path/to/Singularity/install
source Singularity/cvmfs/sphenix.sdcc.bnl.gov/x8664_sl7/opt/sphenix/core/bin/setup_local.sh $MYINSTALL
\end{verbnobox}  
\end{tcolorbox}

\end{enumerate}

\subsection{Jet Analysis}

From inside the Singularity directory, clone the repository \href{https://github.com/ftoralesacosta/e_Jet_sPHENIX}{here}:

\begin{tcolorbox}
\begin{verbnobox}[\scriptsize]
git clone https://github.com/ftoralesacosta/e_Jet_sPHENIX
\end{verbnobox}  
\end{tcolorbox}

Follow the steps from section~\ref{sec:build} to build this package.

\subsection{All-Silicon tracker}

The All-Silicon Tracker we have been studying at LBNL can be found \href{https://github.com/eic/g4lblvtx}{here}.
From inside the Singularity directory, clone the repository:

\begin{tcolorbox}
\begin{verbnobox}[\scriptsize]
git clone https://github.com/eic/g4lblvtx
\end{verbnobox}  
\end{tcolorbox}

Follow the steps from section~\ref{sec:build} to build this package.

% ========================================================================================
\newpage
\section{Running All-Si tracker code}

\subsection{Code details}

The main macro is: \verb|g4lblvtx/macros/Fun4All_G4_FastMom.C|.
This section describes some details from this code.

\subsubsection{Detector geometry}

The detector geometry was created in EICroot, saved into a TGeo file, and then translated into a gdml file. The gdml
file where the detector geometry is defined is: \verb|FAIRGeom.gdml|, and is loaded in the main macro in the lines that read:

\begin{tcolorbox}
\begin{verbnobox}[\scriptsize]
AllSiliconTrackerSubsystem *allsili = new AllSiliconTrackerSubsystem();
allsili->set_string_param("GDMPath","FAIRGeom.gdml");

allsili->AddAssemblyVolume("VST");      // Barrel
allsili->AddAssemblyVolume("FST");      // Forward disks
allsili->AddAssemblyVolume("BST");      // Backward disks
allsili->AddAssemblyVolume("BEAMPIPE"); // Beampipe
\end{verbnobox}
\end{tcolorbox}
	
\subsubsection{Event generator}

\begin{tcolorbox}
\begin{verbnobox}[\scriptsize]
const int particle_gen = 1;
\end{verbnobox}
\end{tcolorbox}

Below are the possible settings for this variable:

\begin{itemize}
\item \verb|particle_gen = 1|, the generator used is a particle generator, which generates a given particle per event over an extended range in angles, momenta, etcetera.
\item \verb|particle_gen = 2|, the generator used is a particle gun, which shoots the same particle with the same properties (angles, momenta, ...) over and over. 
\item \verb|particle_gen = 3|, the generator used is a ``simple event generator'' (using the Fun4All class \verb|PHG4SimpleEventGenerator|) which can generate, for example, 100-pion events.
This cab be used for primary-vertex studies.
\item \verb|particle_gen = 4|, the generator used is pythia8.
\end{itemize}

\subsubsection{Magnetic field}

\begin{tcolorbox}
\begin{verbnobox}[\scriptsize]
const int magnetic_field = 4;   // 1 = uniform 1.5T, 2 = uniform 3.0T, 3 = sPHENIX 1.4T map, 4 = Beast 3.0T map
\end{verbnobox}
\end{tcolorbox}

Below are the possible settings for this variable:

\begin{itemize}
\item \verb|magnetic_field = 1|, uniform 1.5 T solenoidal field.
\item \verb|magnetic_field = 2|, uniform 3.0 solenoidal field.
\item \verb|magnetic_field = 3|, realistic sPHENIX (BaBar) 1.4 T magnetic-field map.
\item \verb|magnetic_field = 4|, realistic Beast 3.0 T magnetic-field map.
\end{itemize}

% ========================================================================================

\subsection{Running All-Si code on login node}

\begin{tcolorbox}
\begin{verbnobox}[\scriptsize]
root -l Fun4All_G4_FastMom.C
\end{verbnobox}
\end{tcolorbox}

or by first loading root, and then doing:

\begin{tcolorbox}
\begin{verbnobox}[\scriptsize]
.x Fun4All_G4_FastMom.C
\end{verbnobox}
\end{tcolorbox}

\subsection{Running batch jobs on Cori}

Some tips to get started with Slurm can be found \href{https://slurm.schedmd.com/quickstart.html}{here}.

\href{https://github.com/eic/g4lblvtx/tree/master/cori_batch}{Here} is an example on how to submit a batch job on Cori.
To submit the jobs from \verb|run_shared_pim.sh| do the following:

\begin{enumerate}
\item exit the container: \verb|exit|
\item go to the directory where \verb|run_shared_pim.sh| is.
\item run: \verb|sbatch run_shared_pim.sh|
\end{enumerate}

To check the status of your jobs, go in your internet browser to \href{https://my.nersc.gov/}{https://my.nersc.gov/} and login.
Then go to the tab ``Cori Queues''.
Alternatively, you can run in the terminal (from outside the Singularity container):

\begin{tcolorbox}
\begin{verbnobox}[\scriptsize]
squeue -u username
\end{verbnobox}  
\end{tcolorbox}

To cancel all submitted jobs, you can run in the terminal (from outside the Singularity container):

\begin{tcolorbox}
\begin{verbnobox}[\scriptsize]
scancel -u username
\end{verbnobox}  
\end{tcolorbox}

%\bibliography{bib_file}
%\bibliographystyle{ieeetr}

\end{document}





